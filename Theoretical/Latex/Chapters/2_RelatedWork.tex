\chapter{Related work}

"God is the one who endowed man with reason, and the mind is the basis of everything. The mind is light and knowledge is a result, and so every knowledge is light" -Jābir ibn Hayyān. This work wasn’t totally completed in one day, and wasn’t completed from beginning to end without the help of online sources. Some of them are in the form of videos, and others are in the form of papers. Most of the sources that will be found here, are more related to the game field to make the graphics of the game and the logic for the player. While the papers have taken the shape of an implemented (Python) library to be used.

On the other hand, the related sources to the field of AI, are implemented as (Python) library to be used. It can be found in papers that require more than average knowledge about basic concepts in this field. I tried to clear most of them in the \hyperref[sec:101-ai]{101 AI} part.

The algorithm itself have been used in other low-budget games with the same purpose of making the environment more interactive, like:

\begin{itemize}
\item Galactic Arms Race (GAR): The game is about a galactic war and there are weapons included as part of the game. "In GAR, all player weapons are generated by the (content-generating NEAT) cgNEAT algorithm based on weapon usage statistics. However, cgNEAT does not simply respawn weapons that people like. Rather, it creates new weapons that elaborate on the ones that have been popular in the past" \citetitle{Galactic_Arms_Race:_Research}.

\item Dance Evolution: is interactive application made by the researcher behind the N.E.A.T algorithm "Kenneth O. Stanley". The game allows you to choose a song, and then a set of dancers will dance to it. When you choose a dancer, it will be the base for others to follow.

\end{itemize}

So as an implementation in the field of games by the creator, there hasn't been a single one that was made recently (in the last 10 years). While there have been little implementation to the idea of having N.E.A.T algorithm to play a game. One is either developed from ground to up like this tutorial \citetitle{Python_Pong_AI_Tutorial_-_Using_NEAT}.  Others have made it on an open source game from another library and add their own tweaks to the library to fit the game, like \citetitle{A.I_Learns_to_Play_Soni_the_Hedgehog_-_NEAT_Explained} where he takes Sonic the Hedgehog module from Gymnasium (formally known as GymAI by openAI) and implement the N.E.A.T algorithm on it.

My implementation will be different from all of the previous examples because:
\begin{enumerate}
\item All the components in the game are pixel-generated, means that there isn't a single image used in the whole game.
\item A full in-depth explanation of the AI is discussed here. As this is the first AI project for me, I elaborate on the details here as if it were my first time learning it. The methodology used in the questions part is the Feynman Technique.
\end{enumerate}



